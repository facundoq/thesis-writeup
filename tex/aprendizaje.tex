\epigraph{The question of whether computers can think is like the question of whether submarines can swim. }{Edsger W. Dijkstra}

%\epigraph{ I believe that something drastic has happened in computer science and machine learning. Until recently, philosophy was based on the very simple idea that the world is simple. In machine learning, for the first time, we have examples where the world is not simple. This means that a good decision rule is not a simple one, it cannot be described by a very few parameters. This is actually a crucial point in approach to empirical inference.}{Vladimir N. Vapnik}



%\epigraph{In classical philosophy there are two principles to explain the generalization phenomenon. One is Occam's razor and the other is Popper's falsifiability. It turns out that by using machine learning arguments one can show that both of them are not very good and that one can generalize violating these principles. There are other justifications for inferences.}{Vladimir N. Vapnik}

%\epigraph{No computer has ever been designed that is ever aware of what it's doing; but most of the time, we aren't either.}{Marvin Minsky}


\section{Introducción}
\texinput{aprendizaje/intro}

\section{Un ejemplo: Reconocimiento de Gestos}
\texinput{aprendizaje/ejemplo}

\section{Aplicaciones}
\texinput{aprendizaje/aplicaciones}

\section{Entrenamiento Supervisado y No Supervisado}
\texinput{aprendizaje/supervisado}

\section{Clasificación}
\texinput{aprendizaje/clasificacion}

\subsection{El Perceptrón}
\label{aprendizaje:perceptron}
\texinput{aprendizaje/perceptron}

\subsection{Experimentos de clasificación}
\texinput{aprendizaje/experimentos}

\subsection{Generalización}
\texinput{aprendizaje/generalizacion}

%\subsection{Validación cruzada}
%\texinput{aprendizaje/validacion}

\subsection{Sobre-especialización y regularización}
\texinput{aprendizaje/regularizacion}

\subsection{Modelo de clasificación con estructura probabilística}
\texinput{aprendizaje/probabilistico}

\section{Resumen}
\texinput{aprendizaje/resumen}
