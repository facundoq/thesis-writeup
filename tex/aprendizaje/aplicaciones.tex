
El ejemplo de los gestos es de un problema de \textbf{clasificación}, pero como mencionamos anteriormente, en general un programa de aprendizaje automático se entrena para aprender una función de inferencia cuya imagen ciertamente no se encuentra restricta a clases de ejemplares. 

Las técnicas de aprendizaje automático se usan con éxito en una enorme variedad de problemas; ya en 1994 se contaba con una gran cantidad de aplicaciones de redes neuronales, uno de los temas de investigación más tradicionales de aprendizaje automático, en las siguientes áreas\cite{widrow1994}:
\begin{itemize}
\item Telecomunicaciones para la mitigación del eco y la ecualización de la señal
\item Control activo para cancelar sonido y la vibración en sistemas como aires acondicionados
\item Detección de eventos y disminución del error de operación en aceleradores de partículas
\item Selección de clientes para la recepción de los catálogos de una empresa
\item Detección de fraudes en operaciones bancarias y tarjetas de crédito
\item Reconocimiento óptico de caracteres de máquina y manuscritos
\item Control de calidad en manufacturas
\item Asignación de asientos en empresas de transporte
\item Marketing
\item Predicción y análisis financiero
\item Control de la manufactura de microprocesadores
\item Control de procesos químicos
\item Detección de cáncer y otras enfermedades en imágenes médicas y muestras biológicas
\item Identificación del origen de drogas ilícitas
\item Exploración y refinación del petróleo
\item Misilísticas
\item Enfocado automático en telescopios
\item Y muchas otras
\end{itemize}

Ampliando algunos de estos casos, en el área de la industria y los servicios, se han utilizado para clasificar el tráfico en conexiones a Internet \cite{yuan2010} y desarrollar sistemas de control integral de manufactura guiados por aprendizaje automático de reglas en base a experiencias previas \cite{shiue2012}. También se han empleado en tareas de predicción de índices de la bolsa \cite{kim2000} y predicción de la disponibilidad de recursos hídricos \cite{maier2000}. Se utilizan en sistemas de control de manufactura, para manejar automáticamente automóbiles, aviones, helicópteros y robots \cite{pomerleau1991,pomerleau1993,azlan1999,rahideh2012}. 

El procesamiento de señales médicas en particular puede verse muy beneficiado por el uso de aprendizaje automático debido a que este suele superar a los métodos estadísticos tradicionales. Se ha utilizado para desarrollar en el uso de marcadores biológicos para la identificación por computadora de cáncer \cite{abeel2010}  y los de en detección de Alzheimer y Trastornos con Déficit de Atención e Hiperactividad \cite{lehmann2007,sidhu2012}, respectivamente. También se han utilizado técnicas de aprendizaje automático en la evaluación de datos neurológicos para resolver diversos problemas, como el diagnóstico clínico de enfermedades, la elaboración de pronósticos y el mapeo de estructuras cerebrales y su funcionamiento. En el área de reconocimiento de patrones cerebrales, se han desarrollado verdaderos juegos mentales \cite{finke2009}, así como interfaces para interactuar con personas en estado vegetativo \cite{lule2012}, con posibles aplicaciones concretas debido a la comercialización de dispositivos de captura en el mercado de consumo.

Tienen gran utilidad en problemas de reconocimiento de voz, del hablante, de objetos en imágenes y videos y de gestos. Su efectividad en estas aplicaciones ha aumentado enormemente en los últimos años, especialmente en el reconocimiento de secuencias de acciones en videos y en etiquetado de escenas \cite{le2011,le2012,farabet2013} y el reconocimiento de voz \cite{hinton2012,deng2013}.

Hay que considerar que varias técnicas de aprendizaje automático como las Redes Neuronales entrenan funciones de inferencia con la propiedad de ser aproximadores universales, es decir, pueden aproximar cualquier función continua en un hipercubo $m$-dimensional $ [0,1]^M $ con un error arbitrariamente pequeño, dado un modelo con la complejidad requerida \cite{haykin1994}, lo cual sienta bases teóricas que apoyan la experiencia de su gran aplicabilidad. Por ende se han utilizado para aproximar funciones como la Transformada Discreta de Fourier \cite{velik2008}, derivar transformaciones similares al Análisis de Componentes Principales No-Lineal (NPCA) \cite{kramer1991} y optimización combinatoria \cite{smith1999}.