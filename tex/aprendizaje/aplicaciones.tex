

Las técnicas de aprendizaje automático se usan con éxito en una enorme variedad de problemas. En un artículo de Widrow \cite{widrow1994} de 1994 ya se contaba con una gran cantidad de aplicaciones de las redes neuronales neuronales artificiales, uno de los temas de investigación más tradicionales de aprendizaje automático. En ellas se mencionan su uso en técnicas de telecomunicaciones para la mitigación del eco y la ecualización de la señal, de detección de eventos y disminución del error de operación en aceleradores de partículas, de detección de fraudes en operaciones bancarias y tarjetas de crédito, reconocimiento óptico de caracteres de máquina y manuscritos, detección de cáncer y otras enfermedades en imágenes médicas y muestras biológicas, y enfocado automático en telescopios.

En la actualidad, la utilización de dichas técnicas no ha menguado. En el área de la industria y los servicios, se han utilizado para clasificar el tráfico en conexiones a Internet \cite{yuan2010} y desarrollar sistemas de control integral de manufactura guiados por aprendizaje automático de reglas en base a experiencias previas \cite{shiue2012}. También se han empleado en tareas de predicción de índices de la bolsa \cite{kim2000} y predicción de la disponibilidad de recursos hídricos \cite{maier2000}. Se utilizan en sistemas de control de manufactura, para manejar automáticamente automóviles, aviones, helicópteros y robots \cite{pomerleau1991,pomerleau1993,azlan1999,rahideh2012}. 

El procesamiento de señales médicas en particular se ha visto muy beneficiado por el uso de aprendizaje automático debido a que este suele superar a los métodos estadísticos tradicionales. Se ha utilizado para desarrollar marcadores biológicos para el diagnóstico por computadora de cáncer \cite{abeel2010}  y para la detección de Alzheimer y Trastornos con Déficit de Atención e Hiperactividad \cite{lehmann2007,sidhu2012}, respectivamente. También se han utilizado técnicas de aprendizaje automático en la evaluación de datos neurológicos para resolver diversos problemas, como el diagnóstico clínico de enfermedades, la elaboración de pronósticos y el mapeo de estructuras cerebrales y su funcionamiento. En el área de reconocimiento de patrones cerebrales, se han desarrollado verdaderos juegos mentales \cite{finke2009}, así como interfaces para interactuar con personas en estado vegetativo \cite{lule2012}, con posibles aplicaciones concretas debido a la comercialización masiva de dispositivos de captura.

Tienen gran utilidad en problemas de reconocimiento de voz, del hablante, de objetos en imágenes y videos y de gestos. Su efectividad en estas aplicaciones ha aumentado enormemente en los últimos años, especialmente en el reconocimiento de secuencias de acciones en videos y en etiquetado de escenas \cite{le2011,le2012,farabet2013} y el reconocimiento de voz \cite{hinton2012,deng2013}.

Hay que considerar que varias técnicas de aprendizaje automático como las Redes Neuronales entrenan funciones de inferencia con la propiedad de ser aproximadores universales, es decir, pueden aproximar cualquier función continua en un hipercubo $m$-dimensional $ [0,1]^M $ con un error arbitrariamente pequeño, dado un modelo con la complejidad requerida \cite{haykin1994}, lo cual sienta bases teóricas que apoyan la experiencia de su gran aplicabilidad. Por ende se han utilizado para aproximar funciones como la Transformada Discreta de Fourier \cite{velik2008}, implementar Análisis de Componentes Principales No-Lineal (NPCA) \cite{kramer1991} y realizar optimización combinatoria \cite{smith1999}.