

Podemos generalizar nuestra función de clasificación de modo que dado un ejemplar $\xv$, $f(\xv)$ devuelva un puntaje por cada clase que represente el grado de pertenencia del ejemplar a la clase. Entonces especificamos $\fdef{f}{\ddp}{\reals_1^{|C|}} $ donde $\reals_1=[0..1] \subset \reals$ y $|C|$ la cantidad de clases. Podemos interpretar dicho puntaje como la estimación del grado de pertenencia del patrón para cada clase, dados los ejemplares de entrenamiento utilizados para generar $f$. Entonces, a cada ejemplar $\xv$ corresponde una etiqueta "verdadera" $\yv \ in \reals_1^{|C|}$ que indica el grado de pertencia a cada clase.
