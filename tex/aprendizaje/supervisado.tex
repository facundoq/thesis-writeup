
Si bien existen varias clasificaciones de los métodos y técnicas de aprendizaje automático, nos centraremos en la más simple, y quizás la más significativa, que distingue algoritmos de entrenamiento \textbf{supervisado} y \textbf{no supervisado}. 

Los algoritmos supervisados son aquellos donde se conoce a priori un resultado que se busca aprender para cada ejemplar de entrenamiento del problema. En el ejemplo de los gestos, el aprendizaje es supervisado si para cada uno de los 20 gestos de la BD ya se conoce su clase, o sea, ya se sabe si son saludos o golpes. El programa entonces aprender a clasificar nuevos gestos en base a etiquetas anteriores.

En el aprendizaje no supervisado no hay o se ignoran los resultados previamente conocidos, no hay etiquetas asociadas a los ejemplares. El programa aprende relaciones entre los datos, descubre las estructuras subyacentes del espacio de ejemplares $\ddp$, pero no se guía por ninguna etiqueta asociada a los ejemplares, sino que usa solamente los datos de los ejemplares mismos. En nuestro ejemplo, sería como tener sólo los 20 gestos pero no saber (al menos a priori) de qué clase son. El programa debería a distinguirlos, a reconocer que los ejemplares de los gestos de saludo tienen cierta similaridad mutua, una estructura compartida; que los egstos de golpe también poseen características similares y, por último, que los gestos de saludo y de golpe son distintos. El programa, dado un nuevo gesto, debe poder entonces decir que es de un tipo o del otro (o ninguno), pero no va a tener codificadas explícitamente las etiquetas \textit{golpe} y \textit{saludo}.

En la práctica, ambas técnicas se utilizan conjuntamente, por lo que si bien podemos tildar a un algoritmo de supervisado o no, en general un sistema completo de aprendizaje automático emplea varias combinaciones de algoritmos de los dos tipos, en forma sucesiva o paralela.

En esta tesis nos enfocamos en técnicas de ML para clasificación de patrones con el objetivo de determinar la clase de gestos a la cual pertenece un ejemplar del mismo representado por una secuencia de posiciones en un espacio $3D$ con etiquetas de tiempo cuya clase es conocida de antemano. Por ende, a continuación describiremos formalmente el problema de la clasificación de ejemplares, enfocándonos en técnicas supervisadas. Además, introduciremos diversos conceptos como generalización, separabilidad lineal, validación cruzada, sobreajuste y regularización para establecer una base teórica común a los métodos de clasificación descriptos en el siguiente capítulo. Si bien por conveniencia tratamos dichos conceptos en el contexto de clasificación, se aplican en general a la mayoría de las técnicas de aprendizaje automático.

