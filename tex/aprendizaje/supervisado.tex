
Si bien existen varias clasificaciones de los métodos y técnicas de aprendizaje automático, la más simple y quizás la más significativa distingue entre algoritmos de entrenamiento \textbf{supervisado} y \textbf{no supervisado}. 

Los algoritmos supervisados son aquellos donde se conoce a priori el resultado que se busca aprender para cada ejemplar de entrenamiento del problema. En el ejemplo de los gestos, el aprendizaje es supervisado si para cada uno de los 20 gestos de la BD se conoce su clase, es decir, se sabe a priori si son saludos o golpes. El programa entonces aprende a clasificar nuevos gestos en base a etiquetas anteriores.

En el aprendizaje no supervisado no hay o se ignoran los resultados previamente conocidos, no hay etiquetas asociadas a los ejemplares. El programa aprende relaciones entre los datos, descubre las estructuras subyacentes del espacio de ejemplares $\ddp$, pero no se guía por ninguna etiqueta asociada a los ejemplares, sino que usa solamente los datos de los ejemplares mismos. En el ejemplo, sería como tener sólo los 20 gestos pero no saber (al menos a priori) de qué clase son. El programa debería aprender a distinguirlos, a reconocer que los ejemplares de los gestos de saludo tienen cierta similitud mutua, una estructura compartida; que los gestos de golpe también poseen características similares y, por último, que los gestos de saludo y de golpe son distintos. El programa, dado un nuevo gesto, debe poder entonces decir que es de un tipo o del otro (o ninguno), pero no va a tener codificadas explícitamente las etiquetas \textit{golpe} y \textit{saludo}.

\imagetwo{supervisado}{0.35}{Esquema de entrenamiento de un algoritmo supervisado.}{no_supervisado}{0.45}{Esquema de entrenamiento de un algoritmo no supervisado.}

En la práctica, ambas técnicas se pueden utilizar conjuntamente, por lo que si bien un algoritmo puede ser tildado de supervisado o no, en general un sistema completo de aprendizaje automático emplea varias combinaciones de algoritmos de los dos tipos.

Esta tesina se enfoca en técnicas de aprendizaje automático para clasificación de patrones con el objetivo de determinar la clase a la cual pertenece un ejemplar de gesto, representado por una secuencia de posiciones en un espacio $3D$, con etiquetas de tiempo cuya clase es conocida de antemano. Por ende, a continuación se describirá formalmente el problema de la clasificación de ejemplares, con énfasis en técnicas supervisadas. Además, se introducirán diversos conceptos como generalización, separabilidad lineal, validación cruzada, sobreajuste (u \textit{overfitting}) y regularización para establecer una base teórica común a los métodos de clasificación descriptos en los siguientes capítulos. Si bien por conveniencia dichos conceptos se tratan en el contexto de clasificación, se aplican en general a la mayoría de las técnicas de aprendizaje automático.

