

\section{Introducción}


A partir de este capítulo se presentarán las decisiones tomadas para resolver el problema de reconocimiento de gestos dinámicos.
El primer paso consiste en establecer un contexto de prueba que permita comparar el método propuesto en esta tesina con otras soluciones existentes en la literatura. Sin embargo, las bases de gestos encontradas se centran más en el problema de captación del gesto que en su reconocimiento \cite{Moni2009,Arpita2013,chaudhary2013}. Por tal motivo, dichas bases suelen estar formadas por gestos muy diferentes entre si facilitando enormemente el proceso de reconocimiento.

En base a lo antes expuesto, para probar los algoritmos se generó una base de datos con gestos de números y letras utilizando un dispositivo Kinect y su SDK, que se denominó Letters and Numbers Hand Gesture (LNHG). En este capítulo se describe la manera en que fue construida dicha base y la manera en que debe ser preprocesada a fin de poder utilizarla como información de entrada a un reconocedor.

A fin de tener un punto de comparación, entre las bases de datos existentes, se seleccionó la definida en \cite{celebi2013}, que se referenciará como Celebi2013, cuyos gestos fueron reconocidos utilizando el algoritmo Dynamic Time Warping presentado en el artículo citado. Es importante remarcar que existen diferencias importantes entre los experimentos realizados en esta tesina y en el artículo \cite{celebi2013}, que serán descriptas oportunamente.


%La base de datos LNHG se generó para esta tesina utilizando un dispositivo Kinect y su SDK. A continuación se describe el proceso de captura, el funcionamiento del Kinect, y las etapas de preprocesamiento y generación de características para cada ejemplar de gesto. 
%

%
%También se presenta una base de datos ajena, que se referenciará como \textbf{Celebi2013} \cite{celebi2013}, para la cual se realizaron experimentos y se obtuvieron resultados comparables al algoritmo de Dynamic Time Warping presentado en el artículo original que dio lugar a la base de datos, si se tienen en cuenta ciertas diferencias importantes entre los experimentos. 


\section{Base de datos de gestos de letras y números arábigos (LNHG)}
\texinput{db/database}

\section{El Kinect y su SDK}
\texinput{db/kinect}

\section{Preprocesamiento}
\texinput{db/normalization}

\section{Características}
\texinput{db/features}

\section{Base de datos de gestos Celebi2013 }
\texinput{db/celebi}

\section{Resumen}
\chaptertexinput{resumen}
