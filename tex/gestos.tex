\epigraph{``Gestures are an elaborate and secret code that is written nowhere, known to none, and understood by all."}{Edward Sapir}

Realizar un gesto es utilizar las distintas partes del cuerpo con el propósito de comunicar información. Mientras que es fácil entender que es un gesto intuitivamente, existen diversos tipos y en general cada tipo involucra un modelado y proceso de reconocimiento con características particulares. A continuación, describimos ciertas clasificaciones y propiedades de los gestos con el objetivo de generar criterio para elegir un modelo específico a reconocer.

El estudio de los gestos tiene una larga historia interdisciplinaria que involucra las disciplinas de la psicología, antropología, linguística, neurociencia, comunicación, actuación, danza, expresión corporal y ciencias de la computación. En la investigación de los gestos, tradicionalmente parte de las ciencias sociales, en general se considera solamente el movimiento de las manos y los brazos en el concepto de gesto y se reserva la palabra kinésica para referirse a la comunicación mediante lenguaje corporal, que involucraría los gestos, el movimiento, la postura, los movimientos de cabeza, la mirada, las expresiones faciales, etc. En la jerga informática, dichas definiciones se mezclan, y se habla de gestos refiriéndose a ambos conceptos.

Existe un gran corpus de investigación sobre la relación entre los gestos y el habla, el pensamiento y los gestos, y la clasificación de gestos. Si bien no existe una taxonomía definitiva para clasificar distintos tipos de gestos, existen ciertos criterios tanto matemáticos como linguísticos para distinguirlos, y algunas clasificaciones establecidas.


\section{Clasificaciones}
\texinput{gestos/clasificacion}


\section{Modelado y propiedades de los gestos dinámicos}
\texinput{gestos/modelo}
