

\subsection{Clasificación de McNeill}

Desde el punto de vista del uso comunicacional de los gestos, es relevante la clasificación de McNeill \cite{mcneill92,Studdert1993}, una de las más conocidas, que surge de experimentos donde la premisa básica era estudiar los gestos que produce una persona que, luego de ver una película, debe contar gestualmente la historia a otra persona que no la ha visto. 

En base a estas y otras experiencias, McNeill propuso la existencia de cuatro tipos principales de gestos: \textbf{icónicos}, \textbf{metafóricos}, \textbf{ilustrativos} y \textbf{deícticos}. Considera que todos son simbólicos, en el sentido en que las partes del cuerpo representan algo diferente que ellas mismas, y que se encuentran estrechamente relacionadas a los aspectos semánticos y pragmáticos del discurso hablado al que acompañan. Si bien en este sentido McNeill se enfoca en estudiar a los gestos como parte de un discurso mayor en el que participan junto al habla y otras expresiones corporales, sus distinciones no dejan de ser relevantes al uso de gestos como método de comunicación e interfaz hombre máquina en ausencia del habla, ya que en ese caso se puede argumentar que existe todavía un discurso, aquel dado por el diálogo entre el actor de los gestos y la interfaz de usuario.

Los \textbf{icónicos} son ``gestos de lo concreto'', representan algún tipo de objeto y dan información acerca de su tamaño, forma, orientación, relaciones espaciales, etc, exhibiendo imágenes transparentes de dicho objeto al que refieren.
Tales imágenes pueden ser redundantes, es decir, co-expresivas, con el discurso: una mano girando en círculos con el dedo índice apuntando hacia abajo significa una torta en una mesa; una mano elevándose representa algo o alguien subiendo. También pueden ser complementarias, capturando un aspecto del discurso que el habla ignora, tal como cuando un narrador describe a una mujer persiguiendo a un perro fuera de su casa e indica su arma - una escoba - no en palabras, sino con movimientos amenazadores del brazo.

Los gestos icónicos tienden a predominar en las narrativas, pero los otros tres, los ``gestos de lo abstracto'', también suelen ocurrir en las narraciones y predominan en otros géneros de interacción como las conversaciones y las exposiciones orales. 

Los gestos \textbf{metafóricos} no son menos pictóricos que los icónicos, pero la imagen que representan es la de una abstracción. Por ejemplo, un tipo de gesto llamado ``conducto metafórico'' representa como algo sustancial el lenguaje, significado, conocimiento, arte u otras nociones abstractas. Tal es el caso cuando un narrador, mientras introduce su obra, levanta las manos como si estuviese sosteniendo una caja y luego las separa, como si dicha caja se rompiese o abriese revelando la historia que contenía dentro. También lo es cuando un hablante dice ``tengo una pregunta'', extendiendo su mano en forma de copa, como esperando recibir una respuesta en ella. 

Los dos tipos de gestos restantes, ilustrativos y deícticos, no son de carácter pictoral. Los \textbf{ilustrativos} son gestos que acompañan a la comunicación verbal para matizar o recalcar lo que se dice, para suplantar una palabra en una situación difícil, etc. Se utilizan intencionadamente. Típicamente son movimientos simples en dos fases (arriba/abajo, adentro/afuera) en donde la mano se mueve de forma rítmica con el habla. Cumplen una función pragmática en un diálogo, indicando que una frase o palabra es importante, pero en general no tienen un contenido semántico de peso.

Finalmente, los gestos \textbf{deícticos} se utilizan para seleccionar o señalar objetos. Son los primeros gestos que realizan los niños para referirse a cosas en su ambiente,  aunque no sólo se utilizan para señalar cosas concretas, como cuando se mueve el brazo para atrás para señalizar el pasado o se indica una dirección. En un ejemplo más elaborado, un narrador puede tomar el espacio delante suyo como el espacio de una obra en donde ubica a sus personajes, y apuntar a ellos como indicación de que las palabras que dice o acciones que realiza son obra de cierto personaje. Es interesante notar que ciertos lenguajes de señas utilizan de forma extensiva los gestos deícticos abstractos, y se presume que ahí tienen su origen.

En el contexto de HCI, los gestos más típicos han sido los deícticos ya que gran parte de la interacción es de carácter espacial, es decir, ir atrás/adelante, entrar, salir, moverse, etc. Los icónicos también son utilizados, por ejemplo, para transmitir el concepto de agrandar o achicar de la función de zoom, para representar caracteres y números gráficamente, y especialmente en juegos en donde generalmente representan acciones tales como utilizar una raqueta de tenis o dar un golpe.
Los gestos ilustrativos y metafóricos no son utilizados en HCI dado que están intrínsecamente relacionados al discurso que acompañan y a relaciones sutiles con el contexto que son muy difíciles de reconocer y traducir en una interfaz.

%AGREGAR:
%
%McNeill (1992) defines three phases of a dynamic gesture: pre-stroke, stroke, and post-
%stroke. Some gestures have both static and dynamic elements, where the pose is important in one or more of
%the gesture phases; this is particularly relevant in sign languages

%\subsection{Clasificación de Cadoz  (SI HAY TIEMPO)}

%described three functional roles of human gesture:
%•  Semiotic – to communicate meaningful information.
%•  Ergotic – to manipulate the environment.
%•  Epistemic – to discover the environment through tactile experience


%\subsection{Clasificación de Kendon (SI HAY TIEMPO)}

%Kendon (1972) described a “gesture
%continuum,” depicted in Figure 2, defining five different kinds of gestures:
%•  Gesticulation. Spontaneous movements of the hands and arms that accompany speech.
%•  Language-like gestures. Gesticulation that is integrated into a spoken utterance, replacing a
%particular spoken word or phrase.
%•  Pantomimes. Gestures that depict objects or actions, with or without accompanying speech.
%•  Emblems. Familiar gestures such as “V for victory”, “thumbs up”, and assorted rude gestures
%(these are often culturally specific).
%•  Sign languages. Linguistic systems, such as American Sign Language, which are well defined.


%\subsection{Clasificación de Caridakis (SI HAY TIEMPO) }
  %\cite{Caridakis2009} ofrece una 

%\subsection{Lenguajes de señas (SI HAY TIEMPO)}

\subsection{Clasificación para su reconocimiento} 

Desde la perspectiva de las ciencias de la computación, se consideran aquellos aspectos de los gestos que impactan en el método de sensado y reconocimiento y en su utilidad como elementos de interacción en una interfaz de usuario o sistema de control. Los más importantes son:

\newcommand{\vs}{\textbf{vs }}

\newcommand{\vsitem}[2]{ \item \textbf{#1 \vs #2} }

\begin{itemize}

\vsitem{ Dinámicos}{Estáticos}
Los gestos \textbf{dinámicos} están formados por una secuencia de movimientos de una o varias partes del cuerpo. Su reconocimiento involucra un modelado espacio-temporal de dichas partes del cuerpo a medida que se realiza el gesto. Es necesaria alguna forma de segmentación para reconocer el comienzo y el fin de los mismos.

Los \textbf{estáticos}, usualmente llamados \textbf{poses}, se representan por una configuración específica de las posiciones de ciertas partes del cuerpo. 

\vsitem{Una sola parte del cuerpo}{Varias}

Los gestos dinámicos realizados con varias partes del cuerpo pueden tener un elemento extra de sincronización en el caso en que algunas partes tarden distinto tiempo en realizar su segmento del gesto o deban comenzar su segmento antes que otras. Si bien son más expresivos, también son más ambiguos y difíciles de realizar.

Los estáticos con varias partes del cuerpo también deben tener en cuenta aspectos temporales para ser reconocidos ya que si bien el gesto es una configuración estática, el movimiento del cuerpo para alcanzar dicha pose es de naturaleza dinámica.

\vsitem{Faciales}{Corporales}

Los \textbf{faciales} involucran el movimiento de los ojos, las cejas y los labios. Están presentes en sistemas de control por movimiento de los ojos, detección de sonrisa en cámaras fotográficas, etc. 

Los \textbf{corporales} involucran el movimiento de distintas partes del cuerpo y se utilizan en sistemas para análisis de movimientos de deportistas, pacientes en rehabilitación médica, entrenamiento de atletas, video juegos, entornos virtuales, etc.

Se pueden considerar aparte también los gestos que involucran el movimiento de los \textbf{dedos de las manos} ya que requieren una precisión mayor y un modelado distinto a los gestos corporales en general.

\vsitem{2D}{3D}

Si bien los gestos obviamente se realizan en un espacio \textbf{3D}, por razones técnicas o de diseño de interfaz en ocasiones se pueden considerar solamente el movimiento en \textbf{dos dimensiones}, en general evitando incluir la de profundidad, ya que resulta más difícil de percibir y controlar que las otras dos.

\vsitem{Unimodal}{Multimodal}

Los sistemas \textbf{unimodales} utilizan una sola fuente de información para realizar el reconocimiento, mientras que los \textbf{multimodales} incorporan información de distintos sensores y diferente naturaleza, como audio, imágenes RGB, imágenes de profundidad, acelerómetros, etc.


\vsitem{Dependientes}{Independientes del usuario}

Al igual que lo que ocurre, por ejemplo, en el reconocimiento del habla, algunos sistemas crean un modelo particular de reconocimiento para cada usuario, quien debe realizar algún tipo de entrenamiento para crear nuevos gestos o adaptar los existentes teniendo en cuenta su forma particular de realizarlos. Estos sistemas se denominan \textbf{dependientes del usuario}. Tienen la ventaja de que en general proveen una performance de reconocimiento más alta y cierta capacidad de personalización al poder agregar nuevos gestos personalizados, con la desventaja de que deben ser entrenados por cada nuevo usuario. Sobre todo, permiten un reconocimiento efectivo para ciertos tipos de gestos complejos en donde hay muchos factores a tener en cuenta, como la contextura del usuario, la velocidad con que se mueve, y las particularidades del lugar donde se encuentra realizando los gestos (en el caso de utilizar una cámara como sensor, su posición respecto de la cámara, las condiciones ambientales de luz). Es apto para aplicaciones donde los gestos se utilizarán con una alta frecuencia, se necesita un gran vocabulario de gestos,  y se espera y necesita una performance muy alta.

Los sistemas \textbf{independientes del usuario} tienen un único modelo de gestos que se comparte para todos los usuarios. Este enfoque tiene la obvia ventaja de que un nuevo usuario no tiene que realizar ningún tipo de entrenamiento para utilizar el sistema, lo cual lo hace apto para interfaces en lugares públicos o de uso ocasional, como un cajero electrónico, o donde se pueden realizar pocas acciones, como en una vidriera interactiva.

\end{itemize}

En esta tesina, el foco está puesto en gestos corporales dinámicos y unimodales, dependientes del usuario, y realizados con una sola parte del cuerpo. Además, se emplea un modelo de gestos basado en el cambio de las posiciones de partes del cuerpo en un espacio 3D a través el tiempo, abstrayéndonos de la manera en que dicha información es capturada (imágenes, acelerómetros, etc). 

