Las redes neuronales artificiales son una familia de modelos computacionales inspirados por las redes neuronales biológicas.


\begin{figure}[ht]
\centering
\begin{minipage}[b]{0.40\linewidth}
\centering
\includegraphics[width=\textwidth]{img/neuronales/biological_mouse}
\caption{Red neuronal biológica, obtenida del área del \textit{giro cingulado} del cerebro de un ratón.}
\label{fig:biological}
\end{minipage}
\hspace{0.5cm}
\begin{minipage}[b]{0.40\linewidth}
\centering
\includegraphics[width=\textwidth]{img/neuronales/artificial}
\caption{Red neuronal artificial. Los círculos representan neuronas y las flechas conexiones entre las mismas.}
\label{fig:artificial}
\end{minipage}
\end{figure}


Desde el punto de vista fisiológico, las neuronas biológicas son células que pueden estimularse eléctricamente. Las neuronas se pueden conectar unas a otras para formar \textbf{redes neuronales}. Los estímulos que genera una neurona son función de los estímulos actualmente recibidos, su estado interno y el tipo y contexto (presión, alargamiento, etc) de la neurona. Los estímulos son mayormente excitativos o inhibitorios. 

Las neuronas son el componente principal del sistema nervioso, que incluye el cerebro, la médula espinal y los ganglios nerviosos periféricos. Se pueden distinguir entonces tres tipos de neuronas principales, en base a la función general que cumplen:

\begin{itemize}
 \item \textbf{Sensoriales}, que responden al tacto, sonido luz y otros estímulos de los órganos sensoriales y los transmiten al cerebro y la médula espinal.
 \item \textbf{Motrices}, que reciben señales del cerebro o la médula espinal y estimulan partes del cuerpo (músculos y glándulas). 
 \item \textbf{Interneuronas} que conectan unas neuronas con otras en la misma región del cerebro o la médula espinal.
\end{itemize}

Desde el punto de vista histológico, las neuronas están compuestas por el cuerpo de la célula, llamado \textbf{soma}, conectores de señales entrantes, llamadas \textbf{dendritas}, y un conector de salida llamado \textbf{axón}. Las dendritas son fibras finas  y cortas que actuan como receptores de impulsos y hay una gran cantidad en cada soma. Hay un solo axón por soma, pero luego de salir de la neurona puede bifurcarse cientos de veces para conectarse con varias dendritas de otras neuronas, y puede tener hasta 1 metro de longitud en los humanos.

\image{histologia}{0.2}{Diagrama de las partes que componen una neurona y su interconexión }

La estimulación entre neuronas ocurre mediante señales electroquímicas, que se propagan a través de las \textbf{sinapsis}, que son conexiones especializadas entre neuronas donde se encuentran ramas del axón de una célula con ramas de las dendritas de otra. La sinapsis no es solamente un simple conector, ya que también modula las señales entre neuronas.

Por supuesto, hay excepciones a estas reglas: neuronas que no tienen dendritas o axón, sinapsis que conectan un axón a otro axón o una dendrita a otra dendrita, etc, pero en general se siguen los patrones recién descriptos.

Si bien la histología de las neuronas se estudia hace tiempo y se conoce con bastante detalle, desde el punto de vista fisiológico todavía queda mucho para descubrir sobre el sistema nervioso. Hay diversos modelos sobre cómo las neuronas cómo grupo o individuos representan y procesan información, cómo influye la arquitectura de las redes en dicha función, cómo se crean y se adaptan a las necesidades del organismo o cómo influye la codificación genética en el desarrollo del cerebro; ninguno de ellos es final, en el sentido de que no existe todavía una teoría completa sobre el funcionamiento del cerebro y el sistema nervioso.

Cualquier modelo neuronal que quiera acercar su comportamiento al de los sistemas biológicos debe contar con una serie de características esenciales que son las que proporcionan a los sistemas biológicos gran parte de sus capacidades.

Una de las principales características de las Redes Neuronales biológicas es su dimensión. En el cerebro existen entre $10^{11}$ y $10^{14}$ neuronas, y cada una de ellas posee entre $1000$ y $10000$ conexiones con otras tantas células. Esto deja a los sistemas artificiales varios órdenes de magnitud por debajo en cuanto a conectividad.

El funcionamiento general de los sistemas nerviosos es altamente paralelo. Es este grado de paralelismo el que permite que, con una velocidad de señal relativamente baja ($1$ mseg) el cerebro pueda reaccionar con rapidez ante la continua necesidad de tomar decisiones, en las que existen un gran número de variables. Dichas decisiones resultan extraordinariamente complejas de plantear, y en muchas casos irresolubles en tiempo real, incluso para los mejores computadores actuales, a pesar de que estos poseen velocidades de alrededor de $6$ órdenes de magnitud superiores a las redes neuronales biológicas. 

En los sistemas nerviosos se realizan procesos de convergencia y divergencia de la información. Mediante la existencia de múltiples conexiones de entrada y salida para cada neurona se produce una multiplicación del uso de una determinada información (divergencia) así como una combinación de diversas señales de entrada para producir un efecto conjunto en una sola neurona (convergencia).

Las redes neuronales biológicas poseen una plasticidad intrínseca, mediante la cual están dotadas de la capacidad de aprender a partir de la experiencia y adaptarse a nuevos entornos con extrema facilidad.

También poseen una cierta predeterminación genética, dada por la existencia de un mapa de conexiones prefijado. Esto significa, básicamente, que el diseño fundamental de la arquitectura de las distintas Redes neuronales está predeterminado genéticamente y ha sido seleccionado de entre los modelos posibles por su valor adaptativo para cada especie. Así pues, la arquitectura general de las conexiones está diseñada de antemano, si bien la potencia relativa de dichas conexiones se obtendrá a lo largo del aprendizaje. Esto no entra en absoluto en contradicción con la idea anterior de plasticidad, como podría en un principio parecer.


En general, los modelos de neuronas artificiales toman en cuenta la idea de neuronas sensoriales, interneuronas y motrices que se llamarán de \textbf{entrada}, \textbf{ocultas} y de \textbf{salida}, respectivamente. Esta correspondencia no es perfecta, ya que algunas neuronas realizan varias funciones, pero es útil en general. También utilizan el concepto de las neuronas y las sinapsis como moduladoras de las señales transmitidas. La mayoría abandona la representación temporal continua por una discreta donde las simulaciones se realizan paso a paso. En la simulación las neuronas actúan de forma sincrónica, o sea, en cada paso de simulación se evalúa la entrada y la salida de cada neurona. Algunas toman en cuenta el tiempo, aunque casi siempre de forma discreta. Además, son de naturaleza digital en lugar de analógica, debido a que se suelen simular con computadoras. 



La gran mayoría de los modelos de neuronas biológicas definen su funcionamiento en términos de ecuaciones diferenciales que describen el cambio de potencial de la misma en función del tiempo y el efecto de otras neuronas, en general basados en el modelo de Huxley-Hodgkin \cite{gerstner2002,burkitt2006}. Las neuronas que modelan son analógicas, de naturaleza continua y generalmente asincrónica, y debido a que consideran el paso del tiempo utilizan el concepto de pulsos  de voltaje generados por la neurona como señal. Dichos modelos luego se extienden para redes enteras, aunque son computacionalmente demandantes, y requieren la estimación de muchos parámetros para generarse \cite{brette2007}. Además, como se mencionó anteriormente, representan aproximaciones al funcionamiento neuronal que todavía tienen mucho por recorrer para asemejarse al comportamiento real de una red neuronal biológica. 

Entonces, si bien se han entrenado redes neuronales artificiales basadas estrictamente en modelos biológicos, en la práctica los modelos artificiales realizan ciertas simplificaciones que los hacen más amenos a la simulación computacional y más efectivos para la resolución de tareas. 

A continuación, se describe un modelo general de red neuronal sincrónica, de tiempo discreto, sin recurrencias ni estado permanente. Luego, se profundiza sobre el modelo de redes neuronales \textbf{feedforward} y dos algoritmos para su entrenamiento, \textbf{backpropagation} (propagación para atrás) y \textbf{resilient backpropagation} (propagación para atrás resiliente), así como el modelo \textbf{competitivo} de redes neuronales y el algoritmo de entrenamiento de \textbf{vector quantization} (cuantización vectorial).


