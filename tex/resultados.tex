\epigraph{``Torture data long enough, and it'll confess to anything"}{}
%All models are wrong, but some are useful. (George E. P. Box)
%"An approximate answer to the right problem is worth a good deal more than an exact answer to an approximate problem." -- John Tukey
%In God we trust. All others must bring data. (W. Edwards Deming)
%Statisticians, like artists, have the bad habit of falling in love with their models. George Box
%The combination of some data and an aching desire for an answer does not ensure that a reasonable answer can be extracted from a given body of data. John Tukey
 %He uses statistics like a drunken man uses a lamp post, more for support than illumination. Andrew Lang


 
\section{Introducción}
 
Este capítulo muestra los resultados de los experimentos realizados con la base de datos Letras y Números (LNHG, por Letters and Numbers Hand Gestures) y la base de datos Celebi2013. También se incluye una discusión sobre la aplicabilidad y efectividad de los métodos propuestos.

El objetivo de estos experimentos es triple: validar el modelo y la característica desarrollada como apta para clasificar gestos, encontrar el clasificador que mejor se adapte a la tarea, junto con sus parámetros ideales, y finalmente validar la hipótesis de que son aptos para generar modelos de gestos con pocas muestras de entrenamiento.


\section{Experimentos con base de datos LNHG}
\label{sec:experimentos}
\chaptertexinput{experimentos}


\section{Resumen}

En este capítulo se comparó el método propuesto en esta tesina en el capítulo 6, denominado CNC, con otros métodos existentes, descriptos en el mismo capítulo.

Los resultados obtenidos permiten afirmar que las capacidades de clasificación del CNC y el modelo SVM son altas, casi alcanzando 90\% de ejemplares correctamente clasificados en la base de datos LNHG, cuando se utilizan conjuntos de entrenamiento pequeños, y que aumenta considerablemente con más gestos de entrenamiento. Los otros modelos funcionan relativamente bien, y algunos hasta de manera comparable al SVM y CNC si se aumenta el tamaño del conjunto de entrenamiento. Estos resultados son bastante robustos a las variaciones de los parámetros de cada algoritmo, indicando que son aptos para el reconocimiento de gestos.
