
Combinando la característica descripta en el capítulo \ref{chap:db} para el modelo del capítulo \ref{chap:gestos}, con los clasificadores de los capítulos \ref{chap:svm} y \ref{chap:neuronales}, y otros desarrollos, es posible construir clasificadores efectivos para el reconocimiento de gestos.

Podemos distinguir dos tipos de características: las globales, que representan alguna característica global del ejemplar, y las locales, que describen el ejemplar en términos de cada parte del mismo, típicamente respetando su topología. En nuestro caso, la característica derivada es del tipo local, ya que cada dirección nos da información local del gesto. De todos modos, pueden generarse características globales en base a ella con la ventaja de que la misma posee las invarianzas ya mencionadas.

A continuación, describimos la aplicación de los modelos SVM y Redes Neuronales Feedforward al problema de clasificación con la característica local desarrollada. Luego, se describe el Clasificador Neuronal Competitivo (CNC), que a partir del vector de direcciones genera un descriptor de tipo global del gesto y lo usa para clasificarlo. Finalmente, describimos un método simple basado en templates que también utiliza la característica desarrollada sin cambios. 

\subsection{Support Vector Machine (SVM)}

Utilizando con la característica directamente, podemos pensar en un gesto como un punto en un espacio n-dimensional de direcciones. Nuestro clasificador toma toda la secuencia de direcciones y genera un modelo que distingue gestos.

En el caso de SVM, asumimos una función de distancia definida entre puntos o gestos de dicho espacio. Dos puntos cercanos en el espacio, en base de esa medida de distancia, se consideran gestos parecidos. Dadas $C$ clases, si entrenamos $C$ clasificadores para que distingan los ejemplares de cada clase de los ejemplares de las otras, encontraremos superficies que separen dichas clases; dichas superficies dependen del kernel elegido. Utilizamos dos kernels en nuestras pruebas: el lineal y el gaussiano. 


En el caso del lineal, estamos simplemente buscando un hiperplano separador. En el caso del kernel gaussiano, podemos notar que la fórmula resultante de $f$ es:

\ma{
f(\xv) &=   \sumi \ai \yi \kn{\xi}{\xv} +b \\
&= \sumi \ai y_i e^{-\norm{\xi-\xv}/\sigma} +b}

donde $\sigma$ es un parámetro de desviación estándar y $\norm{\cdot}$ es la norma euclídea. Entonces, podemos interpretar esta fórmula de la siguiente manera. 

Cada función gaussiana es una función de similitud del ejemplar $\xv$ al ejemplar de entrenamiento $\xi$, escalado por $\sigma$. En el mejor caso $\xi=\xv$, por ende $e^{-\norm{\xi-\xv}/\sigma}=1$ y entonces, el término i-ésimo de la sumatoria es $\ai y_i$. A medida que $\xv$ es cada vez más diferente de $\xi$, $e^{-\norm{\xi-\xv}/\sigma}$ tiende a 0, y entonces dicho término prácticamente se anula en la sumatoria.  

Como $\ai\geq 0 $ y $y_i= \pm 1$, podemos interpretar esto como que se seleccionan los vectores del conjunto de entrenamiento más cercanos a $\xv$, pesados por la distancia entre $\xv$ y $\xi$, y además por la variable dual $\ai$, y se los suma en la dirección en que apunta $y_i$. Si $y_i=1$ el término i-ésimo "vota" para que el vector $\xv$ sea considerado de clase $1$, con una fuerza de votación $\ai  e^{-\norm{\xi-\xv}/\sigma}$, y viceversa si $y_i=-1$. 

Por último, $b$ ajusta la línea de corte, por las mismas razones mencionadas en el capítulo \ref{chap:aprendizaje} para agregar un bias en el perceptrón. En este caso, $\xv$ y $\xi$ son gestos; el kernel gaussiano entonces implementaría nuestra noción del capítulo \ref{chap:gestos} de una equivalencia aproximada $\eequiv$ entre gestos.

\subsection{Redes Neuronales Feedforward (FF)}

Para las redes neuronales feedforward, así como para SVM, utilizaremos el vector de características de cada gesto completo como si fuese un punto en un espacio n-dimensional de gestos.

De todos modos, el vector de direcciones no es simplemente un conjunto de datos, sino que tiene un orden y estructura particular, ya que representa una secuencia ordenada de direcciones, donde bajo la hipótesis de continuidad, cada dirección está correlacionada con la anterior. 

Esta correlación se puede modelar utilizando los modelos de redes neuronales recurrentes, pero también utilizando el modelo \textbf{input-delay} \cite{Haykin1998}, en donde se alimenta a la red con \textit{todo} el vector de direcciones, y se espera que la red misma modele las dinámicas de las secuencias de direcciones. Este es el enfoque que aquí tomamos entonces; la red toma un vector de $n$ direcciones, y por ende utiliza $n$ neuronas de entrada. utilizamos una sola capa de entrada con $h$ neuronas ocultas, determinadas experimentalmente. La capa de salida tiene tantas neuronas como clases, $C$; cada salida $o_c$ representa la confianza de que el ejemplar $\xv$ pertenezca a la clase $c$.
 

\subsection{CNC: El clasificador neuronal competitivo}
\chaptertexinput{cnc}
\subsection{Clasificador basado en templates}
\chaptertexinput{gmm}