El objetivo de esta tesina es estudiar, desarrollar, analizar y comparar distintas técnicas de aprendizaje automático aplicables al reconocimiento automático de gestos dinámicos. Para ello, se definió un modelo de gestos a reconocer, se generó una base de datos de prueba con gestos llamadas \textbf{LNHG} \footnote{Disponible en https://sites.google.com/site/lnhgdb/}, y se estudiaron e implementaron clasificadores basados en máquinas de vectores de soporte (SVM), redes neuronales feedfoward (FF) y redes neuronales competitivas (CPN), utilizando representaciones locales y globales para caracterizar los gestos. Además, se propone un nuevo modelo de reconocimiento de gestos, el clasificador neuronal competitivo (\textbf{CNC}).

Los gestos a reconocer son movimientos de la mano, con invariancia a la velocidad, la rotación, la escala y la traslación. 

La captura de la información referida a los gestos para generar la base de datos se realizó mediante el dispositivo Kinect y su SDK correspondiente, que reconoce las partes del cuerpo y determina sus posiciones en tiempo real. Los clasificadores se entrenaron con dichos datos para poder determinar si una secuencia de posiciones de la mano es un gesto. 

Se implementó una librería de clasificadores con los métodos mencionados anteriormente, junto con las transformaciones para llevar una secuencia de posiciones a una representación adecuada para el reconocimiento \footnote{Todo el código desarrollado en esta tesina se encuentra disponible en https://github.com/facundoq/gest}.

Se realizaron experimentos con la base de datos LNHG, compuesta de gestos que representan dígitos y letras, y con un base de datos de otro autor con gestos típicos de interacción, obteniendo resultados satisfactorios.

