Cuando se considera un ejemplar utilizado para encontrar un hiperplano $(\wv,b)$ en el proceso de optimización y se analiza su posición en el espacio, debe ocurrir alguno de estos tres casos:

\begin{enumerate}
\item $\hyi =1$: El ejemplar está bien clasificado y es un vector de soporte 
\item $\hyi >1$: El ejemplar está bien clasificado y no es un vector de soporte 
\item $\hyi <1$: El ejemplar está mal clasificado
\end{enumerate}


\tikzimage{kkt_casos}{scale=0.3}{Los tres casos posibles en la posición de un ejemplar $\xv$ luego de encontrar el hiperplano de máximo margen en base al valor de $\hyi$.}

Se pueden determinar parcialmente estos casos en base a los valores de $\ai$ y las relaciones:

\begin{equation*}
\aizc \\
\ai+\ui=c  \\
\ai \ri = 0  \\
\ui \si =0 
\end{equation*}

Se pueden dividir también los valores de $\ai$ en tres casos
\begin{equation*}
\ai =0 \\
0 <\ai < c \\ 
\ai = c
\end{equation*}

Y relacionando los tres casos de los valores de las variables $\ai$ con las posiciones de los ejemplares:

\ma{
\ai &< c \tn \ui > 0 \tn \si =0 \tn 
	\begin{cases}
	\ai =0 \tn \riz \tn \text{Caso 1 o 2}\\
	0 < \ai<c  \tn \ri =0 \tn \text{Caso 1}	
	\end{cases} \\
\ai &= c \tn \ui =0 \tn \siz \nd \ri =0 \tn \text{Caso 1 o 3}
}

Pasando en limpio, se sabe que:

\begin{itemize}
\item $\ai =0$ $\tn$ $\xi$ está bien clasificado o es un vector de soporte.
\item $0 <\ai < c$ $\tn$ $\xi$ es un vector de soporte
\item $\ai = c$ $\tn$ $\xi$ está mal clasificado o es un vector de soporte.
\end{itemize}
 
Entonces, si $0 <\aj < c$, se puede utilizar el hecho de que $h(\xj) y_j =1 $ para determinar el valor de $b$. 

