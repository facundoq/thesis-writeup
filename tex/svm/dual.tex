
A continuación, se deriva un modelo alternativo del problema de optimización utilizando la forma dual del mismo. Dicha forma permite, en el caso de SVM, realizar la optimización con mayor eficiencia. De esta forma, dado el problema original, se puede plantear el dual, resolverlo, y luego obtener la solución del problema primal en base al dual.

Todo problema de optimización tiene un problema dual asociado. El problema dual es distinto al problema original o primal, pero están relacionados por las \textbf{condiciones de optimalidad de Karush-Kuhn-Tucker}, que se llamarán simplemente las \textbf{KKT} (ver el apéndice sobre KKT en la sección \ref{sec:kkt} al final del capítulo para una breve referencia sobre el tema.). 

El problema dual se obtiene a partir del primal introduciendo variables duales $\ai$ por cada restricción del problema primal. Si las restricciones son desigualdades, como en el caso de SVM, las variables duales también estarán restrictas como $\ai \geq 0$. Las restricciones se incorporan o codifican en la función de coste, de manera que cuando más se optimizan los términos relacionados a las restricciones en la función, más se cumplen las restricciones. Luego, se eliminan las variables primales (y con ello las restricciones del primal), obteniendo la formulación dual en términos de solamente las variables duales $\ai$. 

En SVM, como la función a optimizar es convexa y las restricciones son afines (y representan un conjunto convexo), cumple las condiciones de dualidad fuerte, lo cual mediante las KKT implica que el valor del mínimo del primal es igual al máximo del dual. Esta relación es muy importante, ya que implica que de cierta manera, minimizar el primal es equivalente a maximizar el dual. Entonces, si se encuentra una relación entre las variables del primal y las del dual que permitan calcular las primeras a partir de las segundas, se puede plantear y resolver el dual, y luego obtener los valores de las variables del primal. 

Para ello, primero se deben llevar las restricciones a la forma $ r \geq 0$. Las de $\si \geq 0 $ ya se encuentren con esa forma; las otras se pueden escribir como $r_i \geq 0$ donde $r_i = \hyi +\si - 1$. 

Luego se introducen variables duales $\ai$, $\rangeD$, una por cada restricción $r_i \geq 0$ y otro conjunto de variables duales $\ui$, $\rangeD$, una por cada restricción $\si \geq 0$, para plantear el comúnmente llamado \textbf{lagrangiano} $\lag$:


\begin{equation*}
\lag(\wv,b,\sv,\av,\uv) = \costew + \costes - \costea - \costeu 
\end{equation*}

donde $\av$ el vector de variables $\ai$ y $\uv$ el vector de variables $\ui$.

El problema queda entonces como \footnote{Por simplicidad, se obvia el rango de la variable $i$ ($\rangeD$) en las siguientes derivaciones}:

\begin{equation*}
\begin{aligned}
&    \minarg{\wv,b,\sv}  \quad
 \{   \maxarg{\av,\uv} \quad  \lag(\wv,b,\sv,\av,\uv)  \} \\
& \text{sujeto a}
&\riz, &\; \siz \\ 
& &  \ai \geq 0, &\; \ui \geq 0 
\end{aligned}
\end{equation*}


El lagrangiano captura el efecto de las restricciones pero en forma de coste, obteniendo un problema equivalente al anterior. La intuición principal detrás de esto radica en qué sucede cuando se cumplen las restricciones $\riz$ y $\siz$. Si $\riz$, entonces $- \costea$ es negativo, lo cual ayuda a minimizar la función de coste. Por lo tanto, para minimizar $\lag$  desde el punto de vista de los nuevos términos en la función de coste, sería ideal que:

\ma{
\ri \gg 0,  \quad \ai \gg 0
}


El mismo argumento vale para $\siz$, $\uiz$ y el término $- \costes$. Por otro lado, si $\riz$ y $\siz$ (lo cual es algo que se necesita en la solución), los términos agregados a la función de coste nunca pueden perjudicarla; a lo sumo, no la ayudarán si $\ai=$ o $\ui =0$. Entonces, estos dos términos nunca aumentan la función de coste si las variables se mantienen dentro de las restricciones. Si no cumplen las restricciones, el coste de la función aumenta, lo cual tendrá el efecto de llevar el proceso de optimización mediante la función de coste a cumplir las mismas. En este balance hay que tener en cuenta que $\wv$ y $\si$ tienen que ser bajos debido a los otros dos términos de la función de coste. En el mejor caso $\si = 0$, y $\hyi \geq 1$, con un $\wv$ bajo. 

Se pueden formalizar estos argumentos intuitivos mediante las KKT (ver el apéndice de la sección \ref{sec:kkt} para un resumen), que además serán utilizadas para eliminar las variables primales y obtener el problema dual. Las KKT dicen que en el óptimo primal y dual $(\wv^*,b^*,\si^*)$ y $(\ai^*,\ui^*)$ se cumplen estas propiedades:


\begin{itemize}

\item Estacionalidad\\
\ma{
\derivative{\lag}{\wv} &= 0 & \;  \derivative{\lag}{b} &= 0 & \; \derivative{\lag}{\si} &= 0
}

\item Feasibilidad Primal	\\
\ma{
\riz & \quad \siz
}

\item Feasibilidad dual\\
\ma{
\aiz & \quad \uiz
}

\item Holgura complementaria\\
\ma{
\ai r_i = 0  \quad \ui \si = 0
}

\end{itemize}

Para justificar formalmente la noción intuitiva de que los términos agregados no cambian fundamentalmente el problema, se utiliza ahora también la propiedad de holgura complementaria; si $\ai r_i=0$ entonces $\costea =0$ en el óptimo, y lo mismo con $\ui \si = 0$ y $\costeu =0$.

Habiendo justificado la equivalencia de los problemas, se utilizará la propiedad de estacionalidad de la solución para encontrar una relación entre las variables duales y primales y eliminar las variables primales del problema, llegando a la formulación dual. Desarrollando las derivadas parciales:

\begin{itemize}
\item \ma{
\derivative{\lag}{\wv} &= \derivative{\, \costew}{\wv} - \derivative{\, \costea}{\wv}  = \wv - \sumayx = 0    \nonumber \\ \tn \wv &=  \sumayx
}
\item \ma{ 
&\derivative{\lag}{b} = \derivative{\costea}{b} = \sumay = 0  \\
&\tn \sumay = 0}

\item \ma{
\derivative{\lag}{\si} &= \derivative{\, \costes}{\si} + \derivative{\, \costeu}{\si} = c - \ai - \ui = 0  \nonumber \\
 \tn \ai  &= c - \ui, \quad \rangeD
} 

\end{itemize}


Reemplazando en $\lag$:

\ma{
\lag = \costew + \costes - \costea - \costeu
}

Como $c = \ai+\ui$

\ma{
  \costes
= \sumi c \si
= \sumi (\ai+\ui)\si
= \sumi \ai \si + \costeu   
}

Reemplazando $\costes$ en $\lag$:

\ma{
  \lag 
= \costew + \sumi \ai \si + \costeu  - \costea - \costeu \nonumber \\
= \costew + \sumi \ai \si - \costea
}


Como:
\ma{
  \costea
= \sumi ((\wv \cdot \xi+b) \yi + \si - 1) \ai \nonumber \\
= \sumi(\wv \cdot \xi) \yi \ai + \sumi  \yi \ai b + \sumi \si \ai - \sumi \ai
}

Dado que $\sumi \yi \ai =0$, entonces $\sumi  \yi \ai b =0$. Entonces


\ma{
  \costea
= \sumi(\wv \cdot \xi) \yi \ai + \sumi \si \ai - \sumi \ai
}

Reemplazando $\costea$ en $\lag$:

\ma{
  \lag 
&= \costew + \sumi \ai \si - \costea \\
&= \costew + \sumi \ai \si - (\sumi(\wv \cdot \xi) \yi \ai + \sumi \si \ai - \sumi \ai)\\
&= \costew + \sumi \ai \si - \sumi(\wv \cdot \xi) \yi \ai - \sumi \si \ai + \sumi \ai \\
&= \costew  - \sumi(\wv \cdot \xi) \yi \ai + \sumi \ai \\
&= \costew  - \wv \cdot \sumayx + \sumi \ai
}

Como $\wv = \sumayx \tn \wv= \sumayx$:

\ma{
  \lag 
&= \costew  - \wv \cdot \sumayx + \sumi \ai \\
&= \costew  - \wv \cdot \wv + \sumi \ai \\
&=  - \costew   + \sumi \ai
}

Para finalmente librarse de la variable $\wv$, se utilizará nuevamente la igualdad anterior, llegando a :

\ma{
\lag 
&=  - \costew   + \sumi \ai \\
&=  - \oh (\sumi \ayx ) \cdot (\suml_j \aj y_j \xj )  + \sumi \ai \\
&=  \lagai
}

Se llamará a esta forma de la función a optimizar $\lag_D$, por Lagrangiano del Dual. De esta manera, $\lag$ no depende de $\wv$, $b$ o $\si$, y por ende las restricciones $\riz$ y $\siz$ pierden efecto, obteniendo el problema equivalente:


\begin{equation*}
\begin{aligned}
&  \maxarg{\av,\uv}
& & \lag_D(\av,\uv)= \lagai \\
& \text{sujeto a}
& & \aiz, \;\;  \uiz, \;  c = \ai+\ui 
\end{aligned}
\end{equation*}

Las restricciones $\aiz$, $\uiz$ y $ c = \ai+\ui $ pueden simplificarse a simplemente $ \aizc$, ya que $\ui$ no aparece en la función de costo ni en otra restricción, obteniendo el problema final 

\begin{equation}
\begin{aligned}
&  \maxarg{\av}
& & \lag_D(\av)= \lagai \\
& \text{sujeto a}
& & \aizc
\end{aligned}
\end{equation}


De nuevo, se puede resolver el problema dual con algún algoritmo genérico de optimización cuadrática (la función es cuadrática por $\ai \aj$), y averiguar el valor de $b$ y $\wv$ como:

\ma{ \wv &=  \sumayx\\
b &= 1 -(\wv \cdot \xv_0) \yi
}

para un vector de soporte $\xv_0$ (ya que para los mismos se sabe que $\wv \cdot \xv_0 +b =1)$. 

Para eso se debe poder determinar en base a los $\ai$ qué ejemplares $\xi$ son vectores de soporte. Además, resulta interesante analizar los valores de $\ai$ ya que darán indicios de si un ejemplar $\xi$ está bien clasificado o no, sin necesidad de calcular $\wv$, $b$ y luego la función de clasificación, como se discute en la siguiente sección. Esta ventaja es necesaria para la eficiencia de ciertos algoritmos iterativos de optimización como SMO, descripto en el apéndice \ref{apendice_smo}. 

