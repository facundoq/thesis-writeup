

La idea esencial, intuitiva, de las SVM es elegir el hiperplano que tenga el mayor margen $M$ entre las dos clases. En otras palabras, la distancia de los ejemplares de entrenamiento más cercanos de cada clase al \textbf{hiperplano de máximo margen} (en adelante, simplemente el hiperplano) es la mayor posible (y es la misma para la dos clases).


\tikzimage{hiperplano_svm_simple}{scale=0.25}{Un conjunto de datos linealmente separable, con un hiperplano de máximo margen.}


Los vectores más cercanos al hiperplano se conocen como \textbf{vectores de soporte}, ya que es su ubicación exacta la que finalmente determina el hiperplano $h$, y los hiperplanos $h^{+}$  y $h^{-}$ que lo contienen se llaman \textbf{hiperplanos de soporte}, paralelos al hiperplano $h$. El margen es la distancia entre un hiperplano de soporte y el hiperplano $h$.

La ecuación del hiperplano es:

\begin{align}
h(\xv) = \hipe{\xv} \quad \wv, \xv \in \reald, b \in \reals
\end{align}

Para SVM, será conveniente codificar las clases $y_i$ como $-1$ y $1$ para representar de que lado del hiperplano están sus ejemplares, como con el Perceptrón.


\tikzimage{hiperplano_svm}{scale=0.25}{Un conjunto de datos linealmente separable, con clases $+$ y $-$. El hiperplano de máximo margen $h(\xv)$ que separa las clases.}


Como se busca un hiperplano que clasifique correctamente a todos los ejemplares $\xi \in D$ \footnote{En este capítulo y el siguiente se denota $D$ al conjunto de entrenamiento, por simplicidad. El enfoque se encuentra en el desarrollo del modelo de clasificación y su algoritmo de entrenamiento, sin considerar la utilización de métodos de evaluación como VC directamente ya que estos pertenecen a la etapa de evaluación del error del clasificador respecto a un conjunto de datos particular.} , se pueden definir las siguientes restricciones:

\begin{align}
\forall \xi \in D \quad \case{ h(\xi) >0 }{h(\xi) <0}{y_i=1}{y_i=-1} 
\end{align}

Como el signo de $y_i$ indica el tipo de desigualdad, se pueden combinar ambos casos como:

\begin{align}
\forall \xi \in D \quad \hyiz
\end{align}

De manera informal, se puede especificar el problema asumiendo una función $M=margen$ que calcule el tamaño del margen entre un hiperplano y un conjunto de ejemplares $D$.

\begin{equation*}
\begin{aligned}
& \minarg{\wv,b}
& & M=margen(\wv,b,D) \\
& \text{sujeto a}
& & \hyiz, \; \rangeD.
\end{aligned}
\end{equation*}

Para formalizar el problema, solo se necesita una expresión concreta de M (el tamaño del margen) en términos de $\wv$ y $b$. 

Pero antes de eso, se debe solucionar un pequeño problema técnico; la ecuación $\hipe{\xv}$ está indeterminada ya que multiplicando por cualquier constante se obtienen valores de $\wv$ y $b$ distintos, pero que representan el mismo hiperplano. Esto da lugar a infinitas soluciones del problema. Entonces, se asume que se buscan los $\wv$ y $b$ con la condición de que para los vectores de soporte de cada clase, $-1$ y $1$, $h(\xvm) =-1$ y $h(\xvp) =1$, respectivamente. De esta manera $\wv$ y $b$ quedan completamente determinados. 

Es importante notar que ahora se pide que $\hyio$  en lugar de $\hyiz$, ya que los $\xi$ más cercanos al hiperplano son los vectores de soporte, para los cuales se cumpliría entonces que $\hyi =1$.

No se escribirá esta restricción explícitamente en la especificación del problema de maximización del margen ya que surgirá implícitamente en la siguiente derivación de una expresión para el margen $M$. Para ello, suponiendo que $\xvp$ es un vector de soporte de la clase $1$, y $\xv_0$ es el vector perteneciente al hiperplano ($h(\xv_0)=0$) más cercano a $\xvp$, tal que la línea que los une es perpendicular al hiperplano, o sea es paralela a $\wv$. 

\tikzimage{margen}{scale=0.3}{Los vectores $\xvp$ y $\xv_0$, pertenecientes al hiperplano de soporte y al hiperplano de máximo margen, respectivamente. La distancia entre los vectores, o sea, la distancia entre los hiperplanos, es el tamaño del máximo margen $M=\lambda \wv$.}

Se puede escribir $\xvp$ como $\xvp = \xv_0 + \lambda \wv$, siendo $\lambda \wv$ el vector que une $\xvp$ y $\xv_0$. La distancia entre $\xvp$ y $\xv_0$, entre el hiperplano y el hiperplano de soporte, es $M=\norm{\lambda \wv}, \, \lambda \in \reals$. Por ende:

\ma{
\wv \cdot \xvp + b &= 1 \\
\wv \cdot (\xv_0 + \lambda \wv) + b &= 1 \\
\wv \cdot \xv_0 + \wv \cdot \lambda \wv + b &= 1 \\
\wv \cdot \xv_0 + b + \wv \cdot \lambda \wv  &= 1 \\
\wv \cdot \lambda \wv  &= 1 \; (\xv_0 \quad \text{pertenece al hiperplano}) \\
\lambda \norm{\wv}^2  &= 1  \\
\lambda \norm{\wv}  &= \frac{1}{\norm{\wv}}  \\
M &= \frac{1}{\norm{\wv}} 
}

Finalmente el problema queda planteado como:


\begin{equation*}
\begin{aligned}
& \maxarg{\wv,b}
& &  \frac{1}{\norm{\wv}}\\
& \text{sujeto a}
& &  \hyiz, \; \rangeD.
\end{aligned}
\end{equation*}


Como maximizar $\frac{1}{\norm{\wv}}$ es equivalente a minimizar $\frac{\norm{\wv}^2}{2} = \costew$ y es más simple, se puede reformular el problema como:

\begin{equation*}
\begin{aligned}
& \minarg{\wv,b}
& &  \costew \\
& \text{sujeto a}
& &  \hyio, \; \rangeD.
\end{aligned}
\end{equation*}

Este es un problema de optimización cuadrática ($\costew$), con restricciones lineales y se puede resolver con algún método de optimización cuadrática y comenzar a utilizar SVM de esa manera. Pero este no es el modelo completo de SVM, ya que es difícil de optimizar y requiere que el problema sea linealmente separable.

A continuación, se extenderá el modelo de SVM con la formulación de \textbf{Márgenes Suaves} para manejar problemas que no son linealmente separables y la presencia de ejemplares anómalos (comúnmente llamados \textbf{outliers}); se derivará la \textbf{formulación dual} que posibilita una solución más simple del problema de optimización y permite el \textbf{truco del kernel} para poder aplicar con poco costo computacional ciertos tipos de transformaciones a los ejemplares para separar mejor las clases. 

Para el lector interesado el profundizar sobre el tema, en el apéndice \ref{apendice_smo} se describe el algoritmo de optimización \textbf{Sequential Minimal Optimization} (SMO) de Platt \cite{platt1998} para resolver el problema de optimización y poder entrenar una SVM.
 