La idea esencial del algoritmo \textbf{Sequential Minimal Optimization} (SMO), en base a la formulación dual, es partir de la solución "nula" $\ai = 0, \; \rangeD$, e ir eligiendo pares de variables duales $(\ai,\aj), \; i \neq j $, optimizando $\lagd$ respecto de estas dos variables, hasta que la solución se haya mejorado al punto que las condiciones KKT se cumplan, con cierta tolerancia. Siendo $\av = (\alpha_1,\cdots,\alpha_D)$ y $kkt$ una función que calcula cuantas variables duales violan las KKT por más que un nivel $\epsilon$, el esquema del algoritmo es:


\begin{algorithm}[H]
\KwData{ \\Un conjunto de ejemplares $D \in \reals^d$\\
Una clase $y_i \in \{-1,1\}$ asociada a cada $\xi \in D$ \\
Una tolerancia a la violación de las KKT, $\epsilon$}
\KwResult{Un hiperplano descripto implícitamente por las variables duales $\av \in \reals^{|D|}$ y un bias $b \in \reals$}
$\av=(0,\cdots,0)$\;
$b=0$\;
\While{$kkt(\av,\epsilon) > 0 $}{ 
  Elegir $\ai$ y $\aj$, $i \neq j$ \;
  Optimizar $\lagd$ respecto de $\ai$ y $\aj$ \;
}
Retornar  $b$ y $\av$
\caption{Esquema del algoritmo SMO} 
\end{algorithm}
\vspace{10pt}

Se pueden considerar tres partes esenciales del algoritmo de manera independiente:

\begin{itemize}
\item En una sección anterior, se analizó la relación que implican las KKT entre los valores de las variables duales ($\ai =0$ , $0 <\ai < c$ y $\ai = c$) con el error del clasificador ($\hyi$). Cuando dicha relación no se mantiene, se detecta una violación de las KKT. De esta forma la función $kkt$ puede contar cuantas variables violan las KKT. La verificación del cumplimiento de las KKT es suave, es decir, se permite cierto error $\epsilon$, típicamente $10^{-2}$ o $10^{-3}$. Verificaciones más estrictas tienen el efecto de retrasar demasiado la convergencia, y no son típicamente necesarias en un clasificador de patrones. 

\item La elección de $\ai$ y $\aj$ se hace prefiriendo aquellas variables duales correspondientes a vectores de soporte que no cumplan las KKT. Primero se elige $\ai$, y luego $\aj$ siguiendo ciertas heurísticas.

\item La optimización de $f$ respecto a $\ai$ y $\aj$ puede hacerse de forma analítica, ya que se considera solo a $\ai$ y $\aj$ como variables y al resto de las variables duales como constantes.
\end{itemize}

Se comenzará primero con la tercera parte del algoritmo, que constituye su esencia. Luego, se lidiará con los primeros dos items que contienen mucho en común, ya que ambos tienen la tarea de ver qué variables no cumplen las KKT y por ende se tratan juntos, mezclando la elección de los $\ai$ y $\aj$ con la condición de fin $kkt(\av,\epsilon) > 0 $. 

\subsection[Optimización de $\lagd$]{Optimización de $\lagd$ con $\ai$ y $\aj$}


Por simplicidad, y sin pérdida de generalidad, en esta sección se asume que las variables duales elegidas para optimizar $\lagd$ son  $\au$ y $\ad$. 

El esquema para buscar una regla de actualización de $au$ y $ad$ que mejore la solución es el siguiente. 

Primero, se llevará $\au$ a la forma $\au = \gamma + s \ad $ ($\gamma$ y $s$ constantes) tomando en cuenta que se está optimizando solo respecto a $\au$ y $\ad$. Utilizando dicha igualdad, se reescribirá $\lagd$ en una expresión $G$ que solo dependa de $\ad$. 

De esa manera, se puede optimizar $G$ (y por ende, $\lagd$) solamente respecto a $\ad$, y después simplemente calcular el valor de la otra. Luego, se derivará $\derivative{G}{\ad}$ para encontrar el máximo. Finalmente, se mostrará como ajustar el $\ad$ y $\au$ considerando las restricciones relevantes sobre $\au$ y $\ad$ provenientes de las KKT, y se calcula el nuevo valor de $\au$.

Los detalles de la derivación para encontrar la expresión $G$ se pueden encontrar en la sección $\ref{sec:smoregla}$ de este apéndice, y el resultado es:

\ma{
G &= \oh \eta \ad^2 +  (y_2 ( E_1^v - E_2^v) - \adv \eta ) \ad 
}

Donde $\adv$ es el valor viejo de la variable $\ad$ y  $\ad$ es el nuevo valor, $E_i^v=f_i^v - y_i$ el error cometido por el clasificador con los valores viejos,  $f_i^v$ es la salida para $\xi$ con los viejos parámetros y $\eta= -\kuu +  2 \kud - \kdd$ es una constante.
De acuerdo al apéndice de la sección $\ref{sec:smoregla}$, la expresión $G$ es tal que $G = \lagd + Constante$, y por ende es equivalente a $\lagd$ para optimizar. Derivando respecto a $\ad$ para encontrar el óptimo:

\ma{
\derivative{G}{\ad} &= \eta \ad + y_2 ( E_1^v - E_2^v) - \adv \eta \\ 
\derivativen{G}{\ad}{2} &= \eta
}

Entonces, si $\derivative{G}{\ad}=0$, se obtiene:
\newcommand{\adadvequation}[1]{\ad #1= \adv + \frac{ y_2 ( E_2^v - E_1^v )}{\eta} }

\ma{
0 &= \eta \ad + y_2 ( E_1^v - E_2^v) - \adv \eta\\
\ad &= \frac{- y_2 ( E_1^v - E_2^v) + \adv \eta}{\eta}  \quad (\eta \neq 0)\\
\ad &= \adv + \frac{- y_2 ( E_1^v - E_2^v)}{\eta} \\
\ad &= \adv + \frac{ y_2 ( E_2^v - E_1^v )}{\eta} \\
\adadvequation{&}
}

Armados con esta última ecuación para el caso $(\eta \neq 0)$, que relaciona de forma simple $\ad$ con $\adv$, se procede ahora a buscar una regla de actualización de las variables $\ad$ y $\au$ para mejorar la solución actual.


\subsection{Regla de actualización para $\au$ y $\ad$}

Debe tenerse en cuenta que las variables $\ad$ y $\au$ no pueden tomar cualquier valor posible ya que la nueva solución debe cumplir las condiciones KKT $0 \leq \ai \leq c$ y $\sum_i \ai y_i=0$. A continuación, $L$ y $H$ denotarán los límites inferiores y superiores que puede tomar la variable $\ad$, cuya existencia se asume ya que se derivarán en la sección siguiente. Estos valores límite asegurarían que la nueva solución cumpla las dos restricciones mencionadas si se elige un nuevo $\ad$ tal que $\ad \in [L,H]$ y se actualiza $\au$ con la regla $\au \ass \sigma - s \ad$,  


Por otro lado, la ecuación $\ad= \adv + \frac{ y_2 ( E_2^v - E_1^v )}{\eta}$ sólo tiene sentido si $\eta \neq 0$. Se puede ver que $\eta \leq 0$, ya que:

\ma{
\eta &= -\kuu +2 \kud - \kdd = - (\kuu-2 \kud +\kdd) \\
&= - (\xv_2-\xv_1)^T(\xv_2-\xv_1)= - \norm{\xv_2-\xv_1} \leq 0
}

Por ende deben manejarse dos casos: $\eta < 0$ y $\eta =0$. 

Si $\eta =0$, $G=y_2 ( E_1^v - E_2^v) \ad$ es una función lineal, y entonces su máximo se encuentra en alguno de los dos límites de los valores posibles para $\ad$, $L$ o $H$. La regla sería entonces:
\newcommand{\updatezeta}{\argmax_{ \ad \in \literalset{L,H}} y_2 ( E_1^v - E_2^v) \ad}

\ma{
\ad \ass  \updatezeta
}
   
Si $\eta <0$, $G$ es una función cuadrática en $\ad$. Revisando la ecuación $\adadvequation{}$, el cambio necesario para estar en el máximo respecto a $\ad$ sería:

\ma{
\Delta \ad = \frac{ y_2 ( E_2^v - E_1^v )}{\eta} 
}

Entonces, se puede utilizar la regla de actualización:: 

\ma{\ad \ass \adv+ \Delta \ad}

Esta regla es preliminar porque, nuevamente, falta analizar los valores límites de $\ad$. Asumiendo la existencia de las constantes $L$ y $H$, y dada una función $restrict_{L,H}(x) = min(max(L,x),H)$ que devuelve su argumento restringido al intervalo $[L,H]$, la regla quedaría como:

\ma{\ad \ass \updateneta}

En resumen, la regla final de actualización sería:
\ma{
\ad \ass \case{\updatezeta}{\updateneta}{\eta = 0}{\eta \leq 0}
}


A continuación se derivan expresiones para los límites $L$ y $H$, y se trata el problema de la actualización de las variables $E_i$ al finalizar un paso de optimización exitoso. Luego, se continúa con los dos puntos restantes: la elección de $\ai$ y $\aj$ y la condición de fin del algoritmo.

\subsubsection{Límites para la variable dual $\ad$}

Cualquier nueva solución debe cumplir $0 \leq \ai \leq c$, pero los límites no son simplemente $L=0$ y $H=c$ ya que debe cumplirse  $\au+s \ad = \gamma$ también. Por ende, para elegir un nuevo valor de $\ad$ (y $\au$, en consecuencia) se deben mantener ambas relaciones. S derivarán los límites considerando los casos $s=1$, y $s=-1$.

Si $s=1$, entonces $\auv+\adv=\gamma$. Entonces, si $\ad>\gamma$, como $0 \leq \au$, no se puede hacer negativo a $\au$ para restarle a $\ad$ y entonces no se puede cumplir la restricción.

Por otro lado, si $\ad < \gamma-c$, entonces $\au$ debería ser mayor que $c$ para llegar a $\gamma$. Entonces:

\ma{
s=1 \tn 
\begin{cases} 
L = max(0,\gamma-c) \\
H = min(c,\gamma) 
\end{cases}
}


Si $s=-1$, entonces $\au- \ad = \gamma$. Si $\gamma \geq 0$, entonces $-\ad \leq  \gamma-c$ ya que sino $\au - \ad$ no podría alcanzar a $\gamma$, debido a que $\au \leq c$. Entonces, $c-\gamma \leq \ad$. Si $\gamma <0$, entonces como sólo $\ad$ es negativo, $-\ad \leq \gamma$, o sea $\ad \geq -\gamma$. Por ende:

\ma{
s=-1 \tn 
\begin{cases} 
L = max(0,-\gamma) \\s
H = min(c,c-\gamma) 
\end{cases}
}

Definiendo funciones $pos$ y $neg$ tales que:

\ma{
pos(1)&=1 & neg(1)&=0 \\
pos(-1)&=0 & neg(-1)&=1 
}

y teniendo en cuenta la relación de los segundos argumentos a $max$ y $min$ con el signo de $s$, se pueden combinar ambos casos: 

\ma{
L= max(0,s(\gamma- pos(s) c )) \\
H= min(c,s(\gamma+ neg(s) c ))
}


\subsubsection{Actualización luego de un paso de optimización exitoso}

Cuando $\au$ y $\ad$ se cambian con $\Delta \au$ y $\Delta \ad$, se deben actualizar los $E_i$ y $b$ para que reflejen los nuevos valores de las variables duales.

El error $E_i$ es:

\ma{
E_i = h(\xi)-y_i =\wv \cdot \xi + b - y_i = \sumj a_j \kij y_j + b - y_i
}

Se podrían calcular los $E_i$ de esta manera, pero no se conoce el nuevo valor de $b$. Pero si $0 < \au < c$ o $0 < \ad < c$, esto quiere decir que dichas variables, que se llamarán variables de soporte, corresponden a vectores de soporte y por ende $E_1=0$ o $E_2=0$, respectivamente. Entonces se puede despejar $b$ con alguna de las siguientes ecuaciones:

\ma{
b &= -\sumj a_j \kuj y_j + y_1 \; \text{o} \\
b &= -\sumj a_j \kdj y_j + y_2 
}

Luego, se calcula $E_i$ para todas las variables duales, utilizando la fórmula anterior o el hecho de que:

\ma{ 
\Delta E_i = \Delta \au y_1 \kui + \Delta \ad y_1 \kdi + \Delta b
}

Si $\au=0$ o $\au=c$ y $\ad = 0$ o $\ad = c$, entonces no se puede calcular de forma exacta $b$ con $\au$ o $\ad$. Una opción es computar $b$ con $\au$ y $\ad$, y promediar los valores como una forma de aproximar el nuevo $b$. 

\subsection{Elección de las variables $\ai$ y $\aj$ y pseudocódigo del algoritmo}

A continuación, se presenta un pseudocódigo del algoritmo, con tres módulos principales: el procedimiento principal, la función $findAjAndOptimize(i)$ y la función $optimize(i,j)$. El primero selecciona la primer variable dual para optimizar y verifica la condición de fin del algoritmo. La segunda busca una segunda variable dual adecuada para optimizar. La tercera realiza la optimización en base a las variables seleccionadas por los procedimientos previos. En esta parte, se dejan de utilizar las variables $\au$ y $\ad$ como las seleccionadas, y se generaliza la notación utilizando $\ai$ y $\aj$, respectivamente.

\subsubsection{Procedimiento principal}

La elección de las variables duales a optimizar se realiza escogiendo primero un $\ai$ apropiado, y luego un $\aj, \; j \neq i$ en base al $\ai$ seleccionado (siempre que sea posible). El procedimiento principal de SMO es:

\begin{algorithm}[H]
\KwData{ \\Un conjunto de ejemplares $D \in \reals^d$\\
Una clase $y_i \in \{-1,1\}$ asociada a cada $\xi \in D$ \\
Una tolerancia a la violación de las KKT, $\epsilon$}
\KwResult{Un hiperplano descripto implícitamente por las variables duales $\av \in \reals^{|D|}$ y un bias $b \in \reals$}
global $\av=(0,\cdots,0)$ \footnote{Las variables global están en el scope de los otros procesos, para simplificar el código.}\;
global $b=0$ \;
global $\ve{E}=(-y_1,\cdots,-y_n)$ \;
changed = false \;
examineAll = true \;
\While{changed OR  examineAll}{ 
  changed= false \;
  \eIf{ examineAll}{
  	randomOffset=random(1,$|D|$) \;
  	\For{index=1 to $|D|$}{
  		i= (index + randomOffset) \% $|D|$ \;
  		changed = changed OR findAjAndOptimize(i) \;
  	}
  }{
	  randomOffset=random(1,$|D|$) \;
	  \For{i=1 to $|D|$}{
	    i= (index + randomOffset) \% $|D|$ \;
	    \If{$ 0 < \ai < c$}{
	      changed = changed OR findAjAndOptimize(i) \;
	    }  
	  }
  }
  
  examineAll = NOT (changed OR examineAll) \;
}
Retornar  $b$ y $\av$ \;
\caption{Procedimiento principal del algoritmo SMO} 
\end{algorithm}
\vspace{10pt}

Donde la función $findAjAndOptimize(i)$ \textit{intenta} optimizar $G$ utilizando la variable $\ai$ y devuelve $true$ si pudo hacerlo o $false$ de lo contrario. 

El algoritmo alterna entre intentar optimizar con las variables de soporte ($0<\ai<c$) e intentar con todas. La razón por la cual se prefiere utilizar las variables de soporte es que son las que definen realmente el hiperplano, y por ende cambiarlas brinda una chance mayor de converger rápidamente. 

Se alterna entre estas dos estrategias ya que llamar a $findAjAndOptimize(i)$ con una variable no implica que se optimizará la función respecto a esa variable (ver la definición de $findAjAndOptimize(i)$) y puede que no haya variables de soporte (por ejemplo, al principio cuando todas las variables duales son $0$).

El recorrido por las variables duales se realiza comenzando desde una posición aleatoria del vector de variables duales $\av$ para evitar sesgar al algoritmo a favor de las primeras variables.

La variable $changed$ se utiliza para detectar cuando no se pudo optimizar respecto a ninguna variable dual. Esto implica, según la definición de los otros módulos, que todas las variables cumplen las $kkt$ a un nivel $\epsilon$, condición de fin del algoritmo. 

\subsubsection{Función $findAjAndOptimize(i)$}

Dado el primer $\ai$, se busca un $\aj$ apropiado. Antes de eso,se debe verificar que $\ai$ no cumpla las KKT. Para ello, se define $R_i=y_i E_i$, una variable fácil de computar, y se reescriben las KKT en términos de $R_i$:

\ma{
\ai=0 &\tn 0 \leq R_i \\
0< \ai<c  &\tn  R_i \simeq 0 \\
\ai=c &\tn R_i \leq 0 
}

que, se pueden escribir más sintéticamente como:

\ma{
(0<\ai \wedge R_i > 0) \vee (\ai<c \wedge R_i < 0) &\tn \text{$\ai$ no cumple las KKT}
}

Introduciendo una tolerancia, queda como:


\ma{
(0<\ai \wedge R_i > \epsilon) \vee (\ai<c \wedge R_i < -\epsilon) &\tn \text{$\ai$ no cumple las KKT}
}


Si $\ai$ viola dichas condiciones, se procede a buscar el $\aj$ apropiado. Se utiliza una heurística que intenta tres cosas en sucesión:
\begin{itemize}
\item Primero, se busca un $\aj$ que sea una variable de soporte, y tal que $|E_i-E_j|$ sea lo más grande posible. Dado que $\Delta \aj = y_j (E_2-E_1) / \eta$, esta heurística intenta que el cambio $\Delta \aj$ sea lo más grande posible para que el algoritmo converja rápidamente.
\item Si el primer criterio falla, se busca un $\aj$ que sea una variable de soporte, por las mismas razones esgrimidas para elegir $\ai$.
\item Por último, se intenta con cualquier $\aj$.
\end{itemize}

De nuevo, seleccionar un $\aj$ con estos criterio no significa que con el par $(\ai,\aj)$ se pueda optimizar $G$, ya que puede suceder que $L=H$ o que $\Delta \aj$ sea muy pequeño, con lo cual el algoritmo no hace realmente un paso de optimización. Por último, en esta búsqueda también se comienza siempre desde una posición aleatoria del vector para no sesgar el algoritmo.

En pseudocódigo:

\begin{function}[H]
\textbf{function} findAndOptimize(i) \;
	$R_i = y_i * E_i$ \;
	\If{($0<\ai$ OR $R_i > 0$) AND ( $\ai<c$ OR $R_i < 0$)}{
		return false \;
	}
  \If{ $\exists \aj: \quad 0 <\aj < c $}{
  	randomOffset=random(1,$|D|$) \;
  	max=0 \;
  	\For{index=1 to $|D|$}{
  		j= (index + randomOffset) \% $|D|$ \;
  		\If{ $0 < \aj<c$ AND $|E_i - E_j|>=max $}{
  		 	max = $|E_i - E_j|$ \;
  		 	jmax= j \;
  		}
  	}
 		\If{optimize(i,jmax)}{
 		  			return true\;
 		}
  }
  randomOffset=random(1,$|D|$) \;
 	\For{index=1 to $|D|$}{
 		j= (index + randomOffset) \% $|D|$ \;
 		\If{ $0 < \aj<c$ AND optimize(i,j)}{
 			return true\;
 		} 
 	}
 	randomOffset=random(1,$|D|$) \;
 	\For{index=1 to $|D|$}{
 		j= (index + randomOffset) \% $|D|$ \;
 		\If{ optimize(i,j)}{
 			return true\;
 		} 
 	}
  return false;
\caption{findAndOptimize(i) } 
\end{function}
\vspace{10pt}

\subsection{Función $optimize(i,j)$}

Finalmente, se desarrolla el proceso de optimización de $\lagd$ respecto a $\ai$ y $\aj$. Primero, se calculan las variables relevantes: $H$, $L$, $s$, $\gamma$, $\eta$, $\Delta \aj$, $\Delta \ai$, luego el nuevo valor de las variables $\ai$ y $\aj$, y finalmente el nuevo valor de $b$ y los $E_i$. Antes de eso, se verifica que el cambio a realizar sea significativo, como ya fue mencionado. Para ello, se necesita que $\Delta \aj / \aj$ sea más grande que un valor pequeño $\epsilon'$, a definir.


\begin{function}[H]
\textbf{function} optimize(i,j) \;
	Calcular $H$, $L$, $s$, $\gamma$, $\eta$, $\Delta \aj$, $\Delta \ai$ según las fórmulas dadas anteriormente \;
	\If{($L \geq H$ OR $\Delta \aj / \aj < \epsilon$) }{
		return false \;
	}
	Actualizar $\ai$ y $\aj$ \;
	Actualizar $b$ \;
	Actualizar $\ve{E}$ \;
	return true \;
\caption{optimize(i,j) } 
\end{function}
\vspace{10pt}


Los cálculos y actualizaciones se hacen en base a las fórmulas de las secciones anteriores, y para un pseudocódigo más detallado se puede consultar el artículo original de Platt \cite{platt1998}.

\subsection{Derivación de la regla de actualización de SMO}
\texinput{svm/smoregla}
\label{sec:smoregla}