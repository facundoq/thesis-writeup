El modelo anterior asume que las clases son linealmente separables, lo cual no suele cumplirse, y además es muy sensible a la presencia de ejemplares anómalos debido a que probablemente se conviertan en vectores de soporte si se encuentran entre las dos clases pero no son representativos de ninguna.


\tikzimagetwo{../aprendizaje/clases_no_linealmente_separables}{scale=0.25}{Un problema no linealmente separable}
{../aprendizaje/outlier_no_razonable}{scale=0.25}{ Un outlier se convierte en vector de soporte y distorsiona el hiperplano de máximo margen \textit{intuitivo}.}

Una manera de lidiar con esto es mediante la extensión \textbf{Soft Margin} (de Márgenes Suaves) de SVM. Dicha extensión consiste en hacer posible que algunos de los ejemplares de entrenamiento no cumplan con la restricción $h(\xi) y_i \geq 1$ dándoles un poco de ayuda mediante una variable de holgura $\si \geq 0$, tal que:

\ma{
\hyi + \si \geq 1,\; \si \in \reals
}


La variable $\si$ ``ayuda'' a $h(\xi) yi$ a llegar a 1 cuando este no puede debido a elección de $\wv$ y b. El problema quedaría entonces como:


\begin{equation*}
\begin{aligned}
& \minarg{\wv,b,\sv}
& &  \lagp(\wv,b,\sv)= \costew\\
& \text{sujeto a}
& &  \hyi + \si \geq 1, \; & \rangeD\\ 
& & & \si \geq 0, \; & \rangeD
\end{aligned}
\end{equation*}


Ahora, si se deja así el problema, los $\si$ pueden tomar valores arbitrarios de manera que las restricciones se cumplirían trivialmente haciendo suficientemente grandes los $\si$. De hecho, se podría definir $\wv = \zv$, ya que de los $\si$ adecuados permitirán que se cumplan las restricciones. Para evitar esto, se penaliza en la función a optimizar cada unidad de $\si$ que se utiliza con un costo $c$, dando lugar a la formulación \footnote{Se eligió usar el mismo coste $c$ para todas las restricciones; si existe información a priori del dominio del problema se puede elegir un coste distinto para cada ejemplar, por ejemplo, o para cada clase, como se suele hacer cuando las mismas están desbalanceadas}:


\begin{equation*}
\begin{aligned}
& \minarg{\wv,b,\sv}
& &  \lagp(\wv,b,\sv)= \costew + \costes \\
& \text{sujeto a}
& &  \hyi + \si \geq 1, \; & \rangeD\\ 
& & & \si \geq 0, \; & \rangeD
\end{aligned}
\end{equation*}

donde $\sv$ es el vector de variables $\si$.

Ahora se tienen dos conjuntos de restricciones: las originales y las de no-negatividad de las variables de holgura.

Dado un $\wv$ y un $b$, es necesario que $\hyio$. Si esto es cierto, se puede tener $\si=0$ y no pagar ningún coste, ya que la restricción se cumple sola. Si no es cierto, existen 2 opciones, en términos de la función a optimizar:  cambiar $\wv$, lo cual puede hacer que otras restricciones no se cumplan y además puede incrementar el coste si se achica el margen, o ayudar a $\hyi$ con $\si$ para que llegue a 1, y pagar el precio $c$. El vector $\xi$ quedaría mal clasificado, en este último caso, pero en términos de la función a optimizar eso puede ser mejor que pagar el costo de achicar el margen. 

Entonces, el proceso de optimización, además de poder cumplir las restricciones si el problema no es linealmente separable, buscará un balance entre una separación perfecta ($\si=0$) y el tamaño del margen ($M \gg 0$), de acuerdo al parámetro de coste $c$. En el caso en que $c \rightarrow \infty$, se penaliza fuertemente la ayuda de $\si$, volviendo al modelo original de márgenes \textit{duros}. Si $c \rightarrow 0$, se permite que $\wv$ y $b$ tomen cualquier valor, porque cualquier error de clasificación puede arreglarse con los $\si$ adecuados ya que son ``baratos''. En el caso extremo, $c=0$, una solución óptima podría ser, por ejemplo, $\wv=0$, $b=0$ y $\si=1$, lo cual es claramente indeseable.


\tikzimage{outlier_soft}{scale=0.25}{Con una elección adecuada de $c$, el problema de optimización converge al hiperplano natural para separar las dos clases, ignorando los outliers.}


El valor de $c$ controla el nivel de regularización de la solución, ya que ayuda a evitar que $\wv$ y $b$ se ajusten exactamente a los datos. Valores altos de $c$ inhiben la regularización, valores muy bajos tienen el efecto contrario al deseado, permitiendo cualquier solución. Por ende, es importante determinar $c$ cuidadosamente con el objetivo de lograr una solución balanceada.


\tikzimagethree{soft_c1}{scale=0.15}{Hiperplano determinado con $c=0.1$. }
{soft_c10}{scale=0.15}{Hiperplano determinado con $c=10$. }
{soft_c1000}{scale=0.15}{Hiperplano determinado con $c=1000$. }


Por último, la función a optimizar $\lagp$ es convexa, lo cual puede verse fácilmente ya que es una suma y multiplicación de funciones convexas, y su dominio es un politopo convexo ya que es la intersección de hiperplanos \cite{boyd2004}. Esto significa que el hessiano es positivo definido, $H(\lagp) \succeq 0$, y que la función tiene un solo mínimo en todo su dominio. Esta simpleza de la función a optimizar contribuye a la elegancia del modelo, ya que hace más ameno el proceso de optimización.