
%drawing



% AXIS

\newcommand{\correctaxiscenter}{
\pgfplotsset{
    standard/.style={
        axis x line=middle,
        axis y line=middle,
        enlarge x limits=0.15,
        enlarge y limits=0.15,
        every axis x label/.style={at={(current axis.right of origin)},anchor=north west},
        every axis y label/.style={at={(current axis.above origin)},anchor=north east}
    }
}
}

\newcommand{\axishundred}{xmin=-100,xmax=100,ymin=-100,ymax=100}
\newcommand{\axisten}{xmin=-10,xmax=10,ymin=-10,ymax=10}
\newcommand{\axisunit}{xmin=-1,xmax=1,ymin=-1,ymax=1}
\newcommand{\axissquare}[1]{xmin=-#1,xmax=#1,ymin=-#1,ymax=#1}
\newcommand{\axisrectangular}[2]{xmin=-#1,xmax=#1,ymin=-#2,ymax=#2}


\newcommand{\drawaxiss}[1]{
\draw [<->] (-#1,0) -- (#1,0);
\draw [<->] (0,-#1) -- (0,#1);
}

\newcommand{\myaxis}[2][]{
\begin{axis}[#1,anchor=origin, x=1cm, y=1cm]  % Align the origins
#2    
\end{axis}
}
\newcommand{\rectangularaxis}[4][]{
\myaxis[\axisrectangular{#3}{#4},#1]{#2}
}


\newcommand{\squareaxis}[3][]{
\myaxis[\axissquare{#3},#1]{#2}
}

\newcommand{\standardaxis}[2][]{
\myaxis[\axisten,#1]{#2}
}

\newcommand{\configureDiscard}{}

\newcommand{\resultsaxis}[3][]{
\correctaxiscenter
\configureDiscard
\pgfplotsset{
    every axis/.append style={font=\Huge},  
}

\myaxis[legend pos=south east, every node near coord/.append style={font=\huge},ylabel=cc,xlabel=#3,#1,ytick={0,25,50,75,100}]{#2}
}

\newcommand{\addplotresults}[6]{
\addplot+[selecting={#1}{#2},#6] table[x index=#3, y index=#4,col sep=comma] {datos/#5};
\addlegendentry{$#1 = {#2}$};
}


%GAUSSIAN

\pgfmathdeclarefunction{invgauss}{2}{%
  \pgfmathparse{sqrt(-2*ln(#1))*cos(deg(2*pi*#2))}%
}
\newcommand{\addplotgaussian}[3]{
\addplot [only marks, samples=#1] ({invgauss(rnd,rnd)+#2},{invgauss(rnd,rnd)}+#3);
}

\newcommand{\addgaussianscatter}[4][30]{
\addplot [only marks, mark=#2,samples=#1] ({invgauss(rnd,rnd)+#3},{invgauss(rnd,rnd)+#4});
}


%aprendizaje

%areas


\newcommand{\classcircle}[3]{\draw [ultra thick, fill={rgb:black,1;white,20}] (#1,#2) circle [radius=#3]
}

\newcommand{\classcirclename}[3]{\classcircle{#1}{#2}{4};
\node [above] at (#1,#2+4) {#3};
}


\newcommand{\classcirclenamelittle}[2]{
\classcircle{#1}{#2}{3.3};
}


\newcommand{\tresclases}{
\classcirclename{0}{4}{Clase 1};
\classcirclename{-5}{-4}{Clase 2};
\classcirclename{5}{-4}{Clase 3};
}


\newcommand{\tresclasessmall}{
\classcirclenamelittle{0}{2.5};
\classcirclenamelittle{-5}{-2.5};
\classcirclenamelittle{5}{-2.5};
}


%scatter plots

\newcommand{\simmetricgaussianscatter}[4][50]{
\addgaussianscatter[#1]{#2}{#3}{#4}
\addgaussianscatter[#1]{#2}{#3}{-#4}
\addgaussianscatter[#1]{#2}{-#3}{-#4}
\addgaussianscatter[#1]{#2}{-#3}{#4}
}


\newcommand{\datosclase}[5][50]{
\addgaussianscatter[#1]{#2}{#3}{#4}
%\addlegendentry{Clase #5};
}


\newcommand{\datostresclases}{
\datosclase{o}{0}{2.5}{1}
\datosclase{triangle}{-5}{-2.5}{2}
\datosclase{diamond}{5}{-2.5}{3}
}         


\newcommand{\datosdosclases}{
\datosclase{oplus}{-5}{2}{+}
\datosclase{halfcircle}{5}{-2}{-}
}

\newcommand{\datosdosclasesnolineal}{
\addgaussianscatter{oplus}{-5}{2}
\addgaussianscatter{oplus}{0}{3}
\addgaussianscatter{oplus}{-5}{-4}
\addgaussianscatter{oplus}{-4}{-1}
\addgaussianscatter{halfcircle}{2.5}{-3}
\addgaussianscatter{halfcircle}{5}{-1}
\addgaussianscatter{halfcircle}{0}{-4}
\addgaussianscatter{halfcircle}{1}{-3.2}
}

\newcommand{\classp}{oplus}
\newcommand{\classm}{halfcircle}
